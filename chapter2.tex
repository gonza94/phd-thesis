\chapter{The FNAL Recycler Ring}

The Fermilab Recycler Ring (RR) is one of the circular accelerators located in the Fermilab Accelerator Complex. It was originally designed to store and accumulate antiprotons that remained from a Tevatron event \cite{rr0}. The recycling of antiprotons was deemed ineffective and was never operationally implemented \cite{rrnagaitsev}. Since 2011, the RR has been repurposed to act as a pre-injector to the Main Injector (MI) by storing and accumulating protons. It is worth pointing out, that the MI and the RR share the same tunnel, which has a circumference of 3.319 km (2.062 mi).\\

The MI/RR complex is fed protons by the Proton Source, which by itself consists of the \cite{pipII1}. The work done for this thesis focuses on the Recycler Ring. Therefore, the following chapter starts by giving a general description for the operation and physics of the Recycler Ring. The next sections introduce and motivate the compensation of third order resonances for high intensity operation.     

\begin{figure}[]
   \centering
   \includegraphics[width=\columnwidth]{chapter1/FNAL.png}
   \caption{The past (Tevatron), present and future (PIP-II and LBNF) of the FNAL Accelerator Complex, taken from [3].}
   \label{fig:fnal}
\end{figure}

\section{General Specifications}

The RR is a permanent magnet storage ring operating at a fixed momentum of 8.835 Gev/c.

\begin{table}[]
\centering
\caption{Typical Recycler Ring properties for beam sent to NuMI}
\label{tab:rrparams}
\begin{tabular}{@{}ccc@{}}
\toprule
\textbf{Parameter}          & \textbf{Value}                             & \textbf{Unit} \\ \midrule
Circumference               & 3319                                       & m             \\
Momentum                    & 8.835                                      & GeV/c         \\
RF Frequency                & 52.8                                       & MHz           \\
RF Voltage                  & 80                                         & kV            \\
Harmonic Number             & 588                                        &               \\
Synchrotron Tune            & 0.0028                                     &               \\
Slip Factor                 & -8.6 $\times$ 10$^{-3}$                    &               \\
Superperiodicity            & 2                                          &               \\
Horizontal Tune             & 25.43                                      &               \\
Vertical Tune               & 24.445                                     &               \\
Horizontal Chromaticity     & -6                                         &               \\
Vertical Chromaticity       & -7                                         &               \\
95\% Normalized Emittance   & 15                                         & $\pi$ mm mrad \\
95\% Longitudinal Emittance & 0.08                                       & eV s          \\
Intensity                   & $5\times10^{10}$, $8\times10^{10}$ (PIP-II) & ppb           \\
MI Ramp Time                & 1.333, 1.133, 1.067                        & s             \\
Booster Frequency           & 15, 20 (PIP-II)                            & Hz            \\ \bottomrule
\end{tabular}
\end{table}

\section{Tune Diagram and Resonances}

\section{High Intensity and Tune Footprint}
