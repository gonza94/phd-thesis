\chapter{Introduction}
\label{sec:ch1}
 
Particle accelerators are the workhorses for modern scientific discoveries. 

\section{Circular Accelerators and Storage Rings}

\section{Fermilab}

\section{Outline}

The following thesis will explore the compensation of third-order resonances in the Fermilab Recycler Ring. This \hyperref[sec:ch1]{first chapter} introduces the motivation behind this thesis work. The \hyperref[sec:ch2]{second chapter} summarizes single particle dynamics with the help of exponential Lie operators and moves forward to introduce a relevant concept of collective beam dynamics: the space charge tune shift. This theoretical overview gives segue into the \hyperref[sec:ch3]{third chapter} of this thesis, where the Recycler Ring is introduced and described in detail. Motivation for the compensation of third order resonances is given in this chapter under the framework of current and future operation of the RR. With the basic physics concepts and the description of the machine put in place, the \hyperref[sec:ch4]{fourth chapter} describes in full detail the scheme and experiments developed in order to compensate third order resonances at low intensities. Before moving to explore the Recycler Ring at high intensities, \hyperref[sec:ch5]{chapter five} provides an interlude in order to show a series of experiments done at the CERN PS Booster. These experiments explore the use of advanced optimization algorithms in the aid of compensating multiple resonance lines simultaneously. Coming back to Fermilab, \hyperref[sec:ch6]{chapter six} showcases the studies and experiments done at high intensities in the RR in order to understand the interplay between the compensation of resonance lines and space charge effects. Finally, \hyperref[sec:ch7]{chapter seven} brings down the curtain by providing some general conclusions and future work stemming from this thesis.