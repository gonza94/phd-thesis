\chapter{Compensation of Third-Order Resonances at Low Intensities}
\label{sec:ch4}

\section{Global RDTs and Lattice Model}

The following chapter explores how to mitigate the effect of third order resonances from the Recycler Ring by means of minimizing the Resonance Driving Terms (RDTs) that drive each resonance. The resonances in question are introduced in Figs. \ref{fig:rrtd} and \ref{fig:rrtdhigh}, and are summarized in Table \ref{tab:rdts}. The RDTs for each of these third order resonance lines can be calculated from Eqs. \ref{eq:rdt1} and \ref{eq:rdt2}. Table \ref{tab:rdts} shows the explicit expression for each third-order RDT of relevance to this work. The sum over $i$, goes through each element of the lattice beam line and asks if it has some sort of sextupole component in its definition---it can be normal $K_{2,i}$ or skew $K_{2,i}^{(s)}$ sextupole component. If it has this multipole, it will add it to the RDT sum by weighting it with the beta functions $\beta_{u,i}$ and phase advances $\phi_{u,i}$ from the linear approximation at those particular locations. Ultimately, the $h_{jklm}$ RDT will be a complex number whose amplitude $|h_{jklm}|$ should be minimized.

\begin{table}[H]
    \centering
    \caption{Corresponding RDTs and spectral lines for each resonance line.}
    \begin{tabular}{lc}
        \toprule
        \textbf{Resonance Line} & \textbf{RDT Expression} \\
        \midrule
            $3Q_x=76$     & $\displaystyle{h_{3000} = -\frac{1}{48}\sum_i K_{2,i} L_i \beta_{x,i}^{\frac{3}{2}} e^{3i\phi_{x,i}}}$    \\ %[3pt]
           $Q_x+2Q_y=74$   &  $\displaystyle{h_{1020} = -\frac{1}{16} \sum_i K_{2,i} L_i \beta_{x,i}^{\frac{1}{2}} \beta_{y,i} e^{i \left[ \phi_{x,i} + 2\phi_{y,i}\right]} }$       \\ %[3pt]
            $3Q_y=73$     &  $ \displaystyle{h_{0030} = -\frac{1}{48}\sum_i K_{2,i}^{(s)} L_i \beta_{y,i}^{\frac{3}{2}} e^{3i\phi_{y,i}}}$ \\ %[3pt]
            $2Q_x+Q_y=75$   & $ \displaystyle{h_{2010} = -\frac{1}{16}\sum_i K_{2,i}^{(s)} L_i \beta_{x,i} \beta_{y,i}^{\frac{1}{2}} e^{i \left[ 2\phi_{x,i} + \phi_{y,i}\right]}}$       \\
        \bottomrule
    \end{tabular}
    \label{tab:rdts}
\end{table}

Figure \ref{fig:h3000bare} shows a visual representation for the calculation of the $h_{3000}$ RDT. This plot shows the amplitude of the complex cumulative sum as it goes around the ring (thick solid orange line). Additionally, this plot also shows the amplitude of each individual contribution for every $i$-th element in the lattice with sextupole component (thin purple line). This particular quantity can be used to visualize where and how the sextupole component is distributed around the ring. Ultimately, after doing this sum around the ring, the final result is a complex number with some amplitude and phase which corresponds to the $h_{3000}$ RDT, as calculated from some arbitrary location in the lattice. The amplitude of the $h_{3000}$ term is plotted in Fig. \ref{fig:h3000bare} with a red dashed line. Figure \ref{fig:h1020bare} shows a similar exercise for the $h_{1020}$ term. All of these calculations are done with a lattice model that has a list of components and magnet coefficients, that, in principle, should be very close to what's inside the tunnel. The particular RR model used was the RR2020V0922FLAT lattice model, provided by R. Ainsworth and M. Xiao.

A question that promptly arises is: does the arbitrary starting position for the sum of Eq. \ref{eq:rdt1} change the RDT result? The short answer is yes, the RDT will change depending on the initial position for the sum. Reference \cite{cernthesis2}, specifically in its Ch. 5, goes into depth as to how to correlate the RDT calculated from a starting point $s_1$ to one measured at starting point $s_2$. The difference in this case relates to the amount of multipole component between both calculation points, e.g., the amount of elements that have sextupole component between an $s_1$ and $s_2$ observation point. Nevertheless, given that there is an infinite amount of $s_1$ and $s_2$ observation points, and only so much real state in this thesis, the plots are for an arbitrary observation point in the lattice. Additionally, given that the sextupole components are evenly distributed around the ring, the RDT values will not oscillate much.

As mentioned in Ch. \ref{sec:ch3}, the Recycler Ring is made up of permanent gradient magnets. From looking at Figs. \ref{fig:h3000bare} and \ref{fig:h1020bare}, one can see that the sextupole component is evenly distributed around some sections of the ring. If one were to plot the distribution of permanent gradient on these plots, the location of them would coincide with the peaks of the individual contributions to the RDT of Figs. \ref{fig:h3000bare} and \ref{fig:h1020bare}. Therefore, the sources that drive the Recycler Ring's normal sextupole resonances come from the permanent magnets themselves---this is known as a systematic-driven resonance as opposed to a random-error-driven resonance. Highly periodic machines and highly linear machines use the fluctuations between RDT measurements from BPMs to locate any sextupole errors in the lattice and try to fix them \cite{cernthesis2}. Nevertheless, this is not the case for the Recycler given its low superperiodicity of 2 and its uniform sextupole component distribution.

\begin{figure}[H]
    \centering
    \includegraphics[width=0.88\columnwidth]{chapter4/h3000_bare.png}
    \caption{Distribution of the $h_{3000}$ term around the ring with individual contributions from each relevant element and the cumulative sum from an arbitrary starting point.}
    \label{fig:h3000bare}
\end{figure}

\begin{figure}[H]
    \centering
    \includegraphics[width=0.88\columnwidth]{chapter4/h1020_bare.png}
    \caption{Distribution of the $h_{1020}$ term around the ring with individual contributions from each relevant element and the cumulative sum from an arbitrary starting point.}
    \label{fig:h1020bare}
\end{figure}

\section{Measurement of Third Order RDTs}

Calculating the theoretical RDTs from the lattice model is a matter of calculating the sums outlined in Table \ref{tab:rdts}. Nevertheless, the measurement of the third order RDTs requires following a long and involved recipe. This recipe is based on previous work from Refs. \cite{cernthesis2,bartolini}, but with a lot of original steps specifically developed for the Recycler Ring. In summary, these are the steps used in order to measure RDTs at the Recycler:
\begin{enumerate}
    \item Kick the beam with dipole kickers and save turn-by-turn BPM data for offline analysis.
    \item Go through data and estimate momentum coordinate with previously calculated transfer matrices from lattice model.
    \item Estimate Twiss parameters and normalized coordinates ($\hat{u}$,$\hat{p}_u$) at every BPM location.
    \item Create resonance basis $h_u^{\pm}$ (see Eq. \ref{eq:hbasis}).
    \item Get spectrum of resonance basis using NAFF (Numerical Analysis of Fundamental Frequencies) through the SUSSIX software \cite{sussix}.
    \item Identify resonance lines from the spectrum that correspond to the RDT of interest.
    \item Calculate the RDT at each BPM location from the equivalence relation of spectral lines and RDT expansion (see Eq. \ref{eq:hx-}). 
\end{enumerate}
The following subsections will explore in more detail the steps outlined in the previous list.

\subsection{Dipole Kick and BPM data}

The first step towards measuring RDTs is to kick (or ping) the beam in one or both transverse direction(s) in order to excite betatron oscillations. Betatron oscillations are the natural oscillations of particles around their equilibrium orbit in a circular accelerator. This kick is done with the help of horizontal and vertical dipole kickers. In particular, the devices used for this were the kicker devices with ACNET names R:K4XXX and R:KVXXX, horizontally and vertically, respectively. In principle, these are the Recycler abort kickers---devices used to send beam to the abort line in Recycler. Nevertheless, the high-voltage settings and timings of these pingers can be changed to give a small kick to the beam. A ping that is small enough not to steer the beam to the abort line, but large enough to excite betatron oscillations in one or both transverse directions. The beam was kicked exclusively in the horizontal direction in order to measure purely horizontal RDTs, e.g., $h_{3000}$, solely in the vertical direction for vertical RDTs, e.g., $h_{0030}$, or pinged in both directions for all of them, including coupling RDTs, e.g., $h_{1020}$ and $h_{2010}$.

Once the kickers are set correctly, the next step is to take BPM data. As mentioned in Sec. \ref{sec:diagnostic}, there are ACNET applications that allow to gather and save BPM data for offline analysis. The BPM data from all 208 BPMs (104 horizontal and 104 vertical) can be saved in one file. Figure \ref{fig:bpm_kick0} shows an instance of kicking the beam in the horizontal direction and recording beam centroid data for 2048 turns at an arbitrary BPM. The amount of turns recorded is also a customizable quantity. The ping shown in Fig. \ref{fig:bpm_kick0} happens early in the cycle, at around 50 turns. The next hundred of turns holds information about the betatron oscillations. This is the data used to extract the tunes $Q_u$ and RDTs of the machine.

\begin{figure}[H]
    \centering
    \includegraphics[width=\columnwidth]{chapter4/bpm_kick.png}
    \caption{BPM data of an arbitrary horizontal kick in the beam at horizontal BPM R:HP620.}
    \label{fig:bpm_kick0}
\end{figure}

The physics of a kicked beam is well explained in Refs. \cite{decoherence1,decoherence2}. Once the beam is pinged, the centroid response will oscillate at betatron tunes $Q_u$. The envelope of these oscillations will be dictated by nonlinearities and chromaticity of the machine. How fast this envelope decays is a measure of the decoherence of the beam. Decoherence of a particle beam in accelerator physics refers to the process by which a beam that initially has particles oscillating in phase—--meaning they have similar amplitudes and frequencies—--gradually becomes out of phase over time. This results in a spread in the particles' positions and momenta, leading to a more diffuse beam. This decoherence is caused by nonlinearities in the machine and the transverse chromaticities that will detune the beam out of coherence, as explained by Eqs. \ref{eq:detune} and \ref{eq:chrom}. Ultimately, this process will diffuse and maim the signal recorded by the BPMs. 

Therefore, when BPM data of kicked beam is taken, special care needs to be taken in order to sustain coherent oscillations. This is done by manipulating the chromaticities of the machine by means of specialized sextupoles. Table \ref{tab:rrparams} showed the nominal chromaticities at which the Recycler Ring operates. In general, for these RDT measurements, the vertical chromaticity was changed between the range of -7 and -3, in order to find sustained vertical oscillations. For the horizontal case, a chromaticity of -5 would be usually enough for 1 mm oscillations. The other system that affects decoherence are the transverse dampers. As per its name, these devices dampen out any oscillation in the beam. Therefore, they were turned off for these particular studies. Taking these factors into account, one could get consequential oscillations in both transverse planes, as a first step to measure Resonance Driving Terms.      

\subsection{Estimation of Momentum Coordinate}

At the most fundamental level, BPM data holds only information about the centroid position of the beam. Nevertheless, in a particle accelerator it is of interest to look at the whole phase space picture ($\hat{u},\hat{p}_u$)---including the momentum coordinate. Therefore, it is of relevance to explain how the momentum coordinate is calculated from the TbT position from every BPM. The approach used involves a model-based perspective. Therefore, the lattice model plays a crucial role in the momentum estimation.    

The approach to estimate the momentum coordinate involves solving a least-squares problem. This is the approach developed in Ref. \cite{yang}. The first step is to calculate the model's transfer matrices (linear approximation) from one fixed point in the accelerator to all the horizontal and vertical BPM locations---in total there should be 208 transfer matrices. For these studies, the fixed point chosen was the starting point of the turn count which is the location for the vertical BPM R:VP601. As an example, let the following paragraphs show how to calculate the horizontal phase space coordinates, including the relative momentum deviation $\delta$. In particular, the main objective is to calculate the $X_0$ matrix with dimensions $3\times2048$, which corresponds to the phase space coordinates $\left( \vec{x}_0, \vec{x}_0', \vec{\delta}_0 \right)$ at the location of R:VP601. The $X_0$ array will have the following definition:
\begin{equation}
    \label{eq:x0vec}
    X_0= 
    \begin{pmatrix}
        \vec{x_0} \\
        \vec{x}_0' \\
        \vec{\delta_0}
    \end{pmatrix} = 
    \begin{pmatrix}
        \left[ x_0(N=1), x_0(N=2), ...,x_0(N=2048) \right] \\
        \left[ x_0'(N=1), x_0'(N=2), ..., x_0'(N=2048) \right] \\
        \left[ \delta_0 (N=1), \delta_0 (N=2), ..., \delta_0 (N=2048) \right]
    \end{pmatrix},
\end{equation}  
and holds the information over the 2048 turns. The least squares problem is defined as the solution to the following system:
\begin{equation}
    \label{eq:lsq}
    A X_0 = B,
\end{equation}
where $A$ is the matrix made from the horizontal coefficients of the model's transfer matrices, and it reads: 
\begin{equation}
\label{eq:alsq}
    A =
    \begin{pmatrix}
    \left( M_{11} \qquad M_{12} \qquad M_{13} \right)_{BPM(i-10)} \\
    \vdots \\
    \left( M_{11} \qquad M_{12} \qquad M_{13} \right)_{BPM(i)}  \\
    \vdots \\
    \left( M_{11} \qquad M_{12} \qquad M_{13} \right)_{BPM(i+10)} 
    \end{pmatrix}.
\end{equation}
The notation $(...)_{BPM(j)}$ means that all the matrix coefficients inside the parenthesis are indexed by the BPM(j), and should be copied from that particular transfer matrix correlating the fixed point to BPM(j). For this case, the BPM(i) corresponds to the particular BPM location where the phase space coordinates $X_{BPM(i)}$ want to be calculated, after calculating $X_0$. It can be noted, that only the 10 upstream BPMs and 10 downstream BPMs of BPM(i) are included in this calculation. This number is easily customizable in this estimation, but should not be too large, i.e., no larger than 30 to preserve some sense of locality. 

The $B$ matrix is defined from the BPM observations, and its transpose is just the BPM responses stacked horizontally. This reads explicitly:
\begin{equation}
    \label{eq:balsq}
    B^T =
    \begin{pmatrix}
    \vdots & & \vdots & & \vdots \\
    \Bigl< x_{BPM(i-10)} \Bigr> & ... &  \Bigl< x_{BPM(i)} \Bigr> &  ... & \Bigl< x_{BPM(i+10)} \Bigr> \\
    \vdots & & \vdots & & \vdots \\
    \end{pmatrix}. 
\end{equation}
For Eq. \ref{eq:balsq}, the triangular bracket notation $\langle ... \rangle$ is used to specify that the BPM data inside the brackets has already been averaged out---the oscillations recorded in the BPM data is centered around 0. Again, in order to use least-squares approximation, this calculation should take the 10 BPMs upstream and the 10 BPMs downstream of the BPM(i), whose phase space coordinates are being calculated. 

The least-squares solution $\hat{X}_0$ to this problem is given by:
\begin{equation}
    \label{eq:x0hat}
    \hat{X}_0 = (A^T A)^{-1} A^T B.
\end{equation}
Once $\hat{X}_0$ is calculated from the data of 10 BPMs upstream and downstream of BPM(i). The phase space coordinates $\hat{X}_{BPM(i)}$ at BPM(i) can be calculated from:
\begin{equation}
    \label{eq:xbpmi}
    \hat{X}_{BPM(i)} = \begin{pmatrix}
        \vec{x} \\
        \vec{x}' \\
        \vec{\delta}
    \end{pmatrix}_{BPM(i)}=
    \begin{pmatrix}
        \left[ x(N=1),...,x(N=2048) \right] \\
        \left[ x'(N=1),..., x'(N=2048) \right] \\
        \left[ \delta (N=1),..., \delta (N=2048) \right]
    \end{pmatrix}
    = \left[ M_{11} \quad M_{12} \quad M_{13} \right]_{BPM(i)} \hat{X}_0.  
\end{equation}
The hat notation $\hat{X}$ is to symbolize that this is data estimated based on the least-squares solution to Eq. \ref{eq:x0hat}. 

Similar to the horizontal case, for the vertical case, the phase space coordinates at each BPM location can be estimated by first estimating the phase space coordinates $Y_0$ at an arbitrary location in the lattice---vertical BPM R:VP601 for this case. At this location, the $Y_0$ array will be defined as:
\begin{equation}
    \label{eq:y0vec}
    Y_0= 
    \begin{pmatrix}
        \vec{y_0} \\
        \vec{y}_0'
    \end{pmatrix} = 
    \begin{pmatrix}
        \left[ y_0(N=1), y_0(N=2), ...,y_0(N=2048) \right] \\
        \left[ y_0'(N=1), y_0'(N=2), ..., y_0'(N=2048) \right]
    \end{pmatrix}.
\end{equation} 
The least-squares estimate $\hat{Y}_0$ for the array in Eq. \ref{eq:y0vec} from the recorded BPM data will be given by:
\begin{equation}
    \label{eq:y0hat}
    \hat{Y}_0 = (A_y^T A_y)^{-1} A_y^T B_y,
\end{equation}
where similar to the horizontal case, the matrix $A_y$ is defined by:
\begin{equation}
    \label{eq:alsqy}
        A_y =
        \begin{pmatrix}
        \left( M_{21} \qquad M_{22} \right)_{BPM(i-10)} \\
        \vdots \\
        \left( M_{21} \qquad M_{22} \right)_{BPM(i)}  \\
        \vdots \\
        \left( M_{21} \qquad M_{22} \right)_{BPM(i+10)} 
        \end{pmatrix},
\end{equation}
and $B_y$ is defined by
\begin{equation}
    \label{eq:balsqy}
    B_y^T =
    \begin{pmatrix}
    \vdots & & \vdots & & \vdots \\
    \Bigl< y_{BPM(i-10)} \Bigr> & ... &  \Bigl< y_{BPM(i)} \Bigr> &  ... & \Bigl< y_{BPM(i+10)} \Bigr> \\
    \vdots & & \vdots & & \vdots \\
    \end{pmatrix}. 
\end{equation}
Once, $\hat{Y}_0$ is calculated, it can be transferred to the location of BPM(i) by means of the transfer matrices. This is the way of calculating $\hat{Y}_{BPM(i)}$, which reads
\begin{equation}
    \label{eq:ybpmi}
    \hat{Y}_{BPM(i)} = \begin{pmatrix}
        \vec{y} \\
        \vec{y}' \\
    \end{pmatrix}_{BPM(i)}=
    \begin{pmatrix}
        \left[ y(N=1),...,y(N=2048) \right] \\
        \left[ y'(N=1),..., y'(N=2048) \right] 
    \end{pmatrix}
    = \left[ M_{21} \quad M_{22} \right]_{BPM(i)} \hat{Y}_0.  
\end{equation}

It is worth pointing out that for the vertical case, the appropriate elements of the transfer matrices are picked out, i.e., $M_{21}$ and $M_{22}$ instead of the horizontal coefficients $M_{11}$, $M_{12}$ and $M_{13}$. The other thing to note is that for the vertical case any momentum dependence is dropped given that the vertical dispersion is negligible in the Recycler Ring, as shown in Fig. \ref{fig:rrdisps}. The previous procedure of calculating $\hat{U}_{BPM(i)}$ is done and saved for each of the 104 horizontal and 104 vertical BPMs---$\hat{X}_{BPM(i)}$ or $\hat{Y}_{BPM(i)}$, accordingly. Figure \ref{fig:momentum} shows an application of this momentum reconstruction technique for a horizontal BPM R:HP620 and its vertical neighbor R:VP621. 

With model-based approaches, it is important to be confident that the model is as close to the real accelerator as possible. The beta-beating is a measure of how well the beta functions of the model describe the beta functions from the real-world accelerator. In particular, M. Xiao has showed that the beta-beating along the Recycler is below 10\% \cite{rr3}---an acceptable quantity for modern accelerators. Therefore, this proves that the model used is reliable up to some significance level. The beta-beating quantity is ultimately limited by how truly linear the accelerator is and any ripple noise from the power supplies feeding the quadrupole and dipoles.

\begin{figure}[H]
    \centering
    \includegraphics[width=\columnwidth]{chapter4/momentum.png}
    \caption{Phase space coordinates reconstruction for two neighboring BPMs---one horizontal R:HP620 and one vertical R:VP621---windowed for 300 turns.}
    \label{fig:momentum}
\end{figure}

\subsection{Twiss Parameters and Normalized Phase Space}

The next step to measure RDTs, once the phase space coordinates have been reconstructed for every BPM, is to build the normal phase space coordinates ($\hat{u}$,$\hat{p}_u$). This is done by using Eqs. \ref{eq:ellipse} and \ref{eq:floquet}, and the information provided in Fig. \ref{fig:ellipses}. In order to build the normalized phase space, first the Twiss parameters have to be estimated. This is done by performing a least-squares fit in order to estimate the ellipse parameters of reconstructed phase space, such as the one shown in Fig. \ref{fig:momentum}. Therefore, an ellipse is being fit to the data, and after that, the Twiss parameters are retrieved from the fit, including the centroid action $2 \pi \langle J_u \rangle =\varepsilon_x $ (see Eq. \ref{eq:ellipse}). 

Figure \ref{fig:ellipse} shows an example of this procedure. The left plot shows reconstructed phase space data with the best ellipse fit. The parameters indicated on the inset provide detailed characteristics of this particular fit: (a) $\beta_x$ (beta function) describes the spatial spread of the beam, (b) $\alpha_x$ (alpha function) relates to the angle the beam particles make with the reference orbit, and it is a measure of the beam divergence, and (c) $\varepsilon_x$ (centroid emittance) quantifies the area of the phase space that the beam centroid occupies. In particular, the centroid action can be calculated from $2 \pi \langle J_u \rangle =\varepsilon_x $.

The right plot of Fig. \ref{fig:ellipse} shows the reconstructed normalized phase space using the Twiss parameters, as estimated from the ellipse fit. This normalization process involves scaling by the square root of the Twiss parameter $\beta_u$ at that particular location, in accordance to Floquet's transformation as defined in Eq. \ref{eq:floquet}. This procedure is done for the reconstructed data of every BPM, and, finally ($\hat{u}$,$\hat{p}_u$) is recorded.  

\begin{figure}[H]
    \centering
    \includegraphics[width=\columnwidth]{chapter4/ellipse_data.png}
    \caption{Reconstructed phase space data with the best ellipse fit (left plot). The right plot shows the reconstructed normalized phase space from using Eq. \ref{eq:floquet} with the Twiss parameters, as estimated from the ellipse fit. This data corresponds to the horizontal data shown in Fig. \ref{fig:momentum} for the R:HP620 BPM.}
    \label{fig:ellipse}
\end{figure}

\subsection{Resonance Basis and Spectral Decomposition}

Once the normalized phase space coordinates are reconstructed for every BPM, the next step is to build the resonance basis $h_u^{\pm}(N)$, as defined in Eq. \ref{eq:hbasis}, i.e., $h_u^{\pm}=\hat{u}\pm \hat{p}_u$. This resonance basis is a function of the number of the turns $N$ that have elapsed. As shown in Eq. \ref{eq:hx-}, this quantity can have a spectral decomposition as a Fourier series. In general, this spectral decomposition in the horizontal dimension reads: 
\begin{equation}
    \label{eq:hxspect}
    h_x^{-}(N)= \hat{x} \pm \hat{p}_x = \sum_{jklm}HSL_{jklm}e^{2\pi i N \left[ \left( 1-j+k\right)Q_x+\left( m-l \right)Q_y\right]},
\end{equation}
and in the vertical dimension:
\begin{equation}
    \label{eq:hyspect}
    h_y^{-}(N)= \hat{y} \pm \hat{p}_y = \sum_{jklm}VSL_{jklm}e^{2\pi i N \left[ \left( k-j\right)Q_x+\left(1-l+m \right)Q_y\right]}.
\end{equation}
These sums run over the $(j,k,l,m)$ indices. In principle, they run all the way to infinity, but can be truncated. The $HSL_{jklm}$ term corresponds to the complex amplitude defining the horizontal spectral line at location $(1-j+k)Q_x+(m-l)Q_y$. The $VSL_{jklm}$ term corresponds to the complex amplitude defining the vertical spectral line at location $(k-j)Q_x+(1-l+m)Q_y$. And as always $Q_u$ corresponds to the betatron tunes. These definitions help to understand the experimental data in order to measure RDTs.

Nevertheless, from the theoretical point of view, the spectral decomposition can be calculated with the Lie algebra gymnastics shown in Sec. \ref{sec:rdts}. These spectral decompositions read for the horizontal plane:
\begin{multline}
    \label{eq:hxpsi2}
    h_x^{-}(N)=\sqrt{2I_x}e^{i\left( \psi_x+\psi_{x_0}\right)} \\
    -2i \sum_{jklm} j f_{jklm} \left( 2I_x \right)^{\frac{j+k-1}{2}}\left( 2I_y \right)^{\frac{l+m}{2}}
    e^{i \left[ \left( 1-j+k\right)\left( \psi_x + \psi_{x_0} \right) +\left( m-l\right)\left( \psi_y + \psi_{y_0} \right)\right]},
\end{multline}
and for the vertical case:
\begin{multline}
    \label{eq:hypsi2}
    h_y^{-}(N)=\sqrt{2I_y}e^{i\left( \psi_y+\psi_{y_0}\right)} \\
    -2i \sum_{jklm} l f_{jklm} \left( 2I_x \right)^{\frac{j+k}{2}}\left( 2I_y \right)^{\frac{l+m-1}{2}}
    e^{i \left[ \left( k-j\right)\left( \psi_x + \psi_{x_0} \right) +\left( 1-l+m\right)\left( \psi_y + \psi_{y_0} \right)\right]}.
\end{multline}
The RDT calculation exploits the equivalence between Eqs. \ref{eq:hxspect} and \ref{eq:hyspect} to Eqs. \ref{eq:hxpsi2} and \ref{eq:hypsi2}. Therefore, the generating function coefficients ($f_{jklm}$) can be related to the horizontal and spectral line coefficients ($HSL_{jklm}$ and $VSL_{jklm}$). Ultimately, the $f_{jklm}$ terms can be related to Resonance Driving Terms $h_{jklm}$.

\subsection{Resonance Basis Spectrum}

\begin{figure}[H]
    \centering
    \includegraphics[width=\columnwidth]{chapter4/hxspect.png}
    \caption{Spectral lines of $h_x^{-}$ calculated with SUSSIX \cite{sussix}. The $h_x^{-}$ signal was reconstructed for the 104 Horizontal BPMs. The spectrum for all BPMs is superimposed in this plot.}
    \label{fig:hxspect1}
\end{figure}

\begin{figure}[H]
    \centering
    \includegraphics[width=\columnwidth]{chapter4/hyspect.png}
    \caption{Spectral lines of $h_y^{-}$ calculated with SUSSIX \cite{sussix}. The $h_y^{-}$ signal was reconstructed for the 104 Vertical BPMs. The spectrum for all BPMs is superimposed in this plot.}
    \label{fig:hyspect1}
\end{figure}

\subsection{Spectral Lines and RDT calculation}

\begin{table}[H]
    \centering
    \caption{Corresponding RDTs and location of spectral lines for each resonance line.}
    \begin{tabular}{lcccc}
        \toprule
        \textbf{Resonance Line} & \textbf{Source} & \textbf{RDT} & \textbf{Hor. Spect.} & \textbf{Vert. Spect.} \\
        \midrule
            $3Q_x=76$     & Normal Sextupole    & $h_{3000}$           &  (-2,0)  & -       \\ %[3pt]
           $Q_x+2Q_y=74$   & Normal Sextupole    & $h_{1020}$            & (0,-2) & (-1,-1)       \\ %[3pt]
            $3Q_y=73$     & Skew Sextupole   & $h_{0030}$           & - & (0,-2)        \\ %[3pt]
            $2Q_x+Q_y=75$   & Skew Sextupole    & $h_{2010}$     & (-1,-1) & (-2,0)       \\
        \bottomrule
    \end{tabular}
    \label{tab:rdtlines}
\end{table}

\subsection{Third Order RDTs at every BPM location}

\section{Compensation of RDTs}

For resonance compensation there are four dedicated normal sextupoles with currents that can be set to ($I_{sc220},I_{sc222},I_{sc319},I_{sc321}$) and four dedicated skew sextupoles with currents that can be set to ($I_{ss323},I_{ss323},I_{ss319},I_{ss321}$). As shown in the previous section one RDT can be cancelled out with the right kick from the correction elements, which means the resonances are corrected to first order. 

Nevertheless, by compensating one resonance line, other resonances  might become worse. This is why for simultaneous compensation, compensation currents will vary depending on the subsets of resonances to compensate. In principle, the currents $I_x$ needed in each correction element in order to cancel out the four bare machine RDTs, are given by the solution to this linear system of equations: 
\begin{equation}
    \begin{bmatrix}
        -|{h_{3000}}|  \cos (\psi_{3000})\\
        -|{h_{3000}}|  \sin (\psi_{3000})\\
        -|{h_{1020}}|  \cos (\psi_{1020})\\
        -|{h_{1020}}|  \sin (\psi_{1020})\\
        -|{h_{0030}}|  \cos (\psi_{3000})\\
        -|{h_{0030}}|  \sin (\psi_{3000})\\
        -|{h_{2010}}|  \cos (\psi_{1020})\\
        -|{h_{2010}}|  \sin (\psi_{1020})\\
      \end{bmatrix}_{(Bare)}
    =
      \boldsymbol{M}
    \begin{bmatrix}
        I_{sc220} \\
        I_{sc222} \\
        I_{sc319} \\
        I_{sc321} \\
        I_{ss223} \\
        I_{ss323} \\
        I_{ss319} \\
        I_{ss321} \\
      \end{bmatrix}
    \label{eq:system}
\end{equation}
where $M_{ij}$ is the response matrix for the RDTs with respect to the currents, and includes any roll that can happen for the correction sextupoles. This response matrix $M_{ij}$ can be calculated by scanning the currents in each correction element and looking at the response from the real and imaginary part of the RDTs, i.e., $h_{jklm}=|h_{jklm}|e^{i\psi_{jklm}}$. 

In reality, there are limitations to solving equation \ref{eq:system}. The first one is that all the RDTs ($h_{jklm}$) may not be accessible for measurement, given that they may not show up as a spectral line.  Another limitation is that the solution for the currents may be outside the maximum limits for the correction elements. 

One can also try to cancel out a subset of RDTs from equation \ref{eq:system}, including only one RDT. For example, in order to compensate $3 Q_x=76$, the system of equations to be solved is:

\begin{equation}
    \begin{bmatrix}
        -|{h_{3000}}|  \cos (\psi_{3000})\\
        -|{h_{3000}}|  \sin (\psi_{3000})\\
        0\\
        0\\
      \end{bmatrix}_{(Bare)}
    =
    \begin{bmatrix}
        M_{11} & M_{12} & M_{13} & M_{14} \\
        M_{21} & M_{22} & M_{23} & M_{24} \\
        1 & 1 & 0 & 0 \\
        0 & 0 & 1 & 1 \\
    \end{bmatrix}
    \begin{bmatrix}
        I_{sc220} \\
        I_{sc222} \\
        I_{sc319} \\
        I_{sc321} \\
      \end{bmatrix}
    \label{eq:system1}
\end{equation}

\begin{figure}[H]
    \centering
    \includegraphics[width=\columnwidth]{chapter4/h3000_3qxcomp.png}
    \caption{Distribution of the $h_{3000}$ term around the ring with individual contributions from each relevant element and the cumulative sum from an arbitrary starting point when correction elements are set to compensate $3Q_x=76$, i.e., $h_{3000}=0$.}
    \label{fig:h3000_3qxcomp}
\end{figure}

\begin{figure}[H]
    \centering
    \includegraphics[width=\columnwidth]{chapter4/h1020_3qxcomp.png}
    \caption{Distribution of the $h_{1020}$ term around the ring with individual contributions from each relevant element and the cumulative sum from an arbitrary starting point when correction elements are set to compensate $3Q_x=76$, i.e., $h_{3000}=0$.}
    \label{fig:h1020_3qxcomp}
\end{figure}

\section{Optimization of Compensation Currents}

\section{Experimental Verification of Compensation}

\subsection{\label{sec:lossmaps}Dynamic Loss Maps}

One effective method for visualizing resonance compensation involves constructing dynamic loss maps. To generate these representations, specialized quadrupoles responsible for controlling the tune of the Recycler are gradually adjusted to map out the desired tune area. Throughout this process, the beam loss rate is meticulously measured and interpolated across the specified region. This is done in the horizontal and vertical direction. The initial horizontal scan is generated by maintaining a constant vertical tune while implementing a horizontal tune ramp ranging from $Q_x=25.47$ to $Q_x=25.31$. Subsequently, the vertical tune, initially set as \textit{constant} at $Q_y=24.47$, is adjusted incrementally to $Q_y=24.31$ in steps of 0.005, with intensity data recorded at each step. Conversely, for the vertical scan, the roles are reversed: the horizontal tune remains constant while a vertical tune ramp progresses from $Q_y=24.47$ to $Q_y=24.31$. Then, the \textit{constant} horizontal tune is varied from $Q_x=25.47$ to $Q_y=25.31$ in steps of 0.005. The resulting intensity data from both scans can be differentiated, normalized by the instantaneous intensity, and interpolated within a two-dimensional grid to construct plots akin to those depicted in Figs. \ref{fig:bare_nocomments} and \ref{fig:bare_comments}. Figure \ref{fig:bare_nocomments} demonstrates the initial machine scan without any compensation. If plotted alongside the theoretical positions of the lines as in Fig. \ref{fig:bare_comments}, the beam loss bands align with the resonance lines. 

Figure \ref{fig:bare_comments} illustrates the correspondence between the loss patterns and the theoretical positions of resonance lines. A slight deviation exists between the set tune and the actual tune due to calibration adjustments from the tune trombone program. However, despite this variance, the resonance line configuration within the loss pattern facilitates the visualization of each resonance's strength. Specifically, a higher normalized loss at a particular tune location indicates a stronger Resonance Driving Term (RDT) for the corresponding resonance line. Within Figs. \ref{fig:bare_nocomments} and \ref{fig:bare_comments}, third, fourth, and even traces of fifth-order resonance lines are discernible, with third-order resonance lines exhibiting the greatest prominence.

Figure \ref{fig:lossmaps} depicts dynamic loss maps representing various configurations of the compensation sextupoles. Specifically, Fig. \ref{fig:sfig1} illustrates the loss map for the bare machine, where no compensation sextupoles are activated, while Fig. \ref{fig:sfig2} and Fig. \ref{fig:sfig3} demonstrate compensation for a single resonance line each. In the case of $3Q_x$ compensation, the four normal sextupoles are adjusted to the calculated compensation currents using the RDT response matrix method. Moreover, a comparison between Fig. \ref{fig:sfig2} and Fig. \ref{fig:sfig1} clearly indicates a reduction of normalized losses by two orders of magnitude at the $3Q_x$ line with compensation. This observation holds true for the $3Q_y$ compensation as well, as shown in Fig. \ref{fig:sfig3}.

Figures \ref{fig:sfig4}, \ref{fig:sfig5}, and \ref{fig:sfig6} showcase the optimal configurations of compensation sextupoles designed to address multiple resonance lines simultaneously. It's important to note that while attempting to compensate for one or multiple resonance lines, there's a possibility that other resonance lines may strengthen. This is evident in the explicit case depicted in Fig. \ref{fig:sfig6}, where compensating for $3Q_y$ and $Q_x+2Q_y$ leads to the amplification of the $2Q_x+Q_y$ resonance. Such occurrences pose a limitation when aiming to compensate for more than two resonance lines, as the compensation currents tend to increase. There exists a constraint on the currents supplied to the compensation sextupoles. For instance, in compensating both normal sextupole lines, $3Q_x$ and $Q_x+2Q_y$, the required currents exceed the current limit. Ongoing efforts are focused on reducing the compensation currents in this specific scenario. Section \ref{sec:addsexts} summarizes some of these efforts, where additional sextupoles have been installed in order to decrease the currents that cancel out both the $h_{3000}$ and $h_{1020}$ RDTs.

Another notable detail evident in Figs. \ref{fig:sfig1}-\ref{fig:sfig6} is the presence of white areas within the loss maps, indicating regions where there was insufficient beam to accurately map out the losses. In certain configurations of the compensation sextupoles, the combined weakening of the third-order resonance lines occurs in a manner that leaves some beam remaining beyond these lines. It could be argued that conducting two additional scans, injecting from the left and bottom, could effectively map out these inaccessible regions. Such an enhancement could be considered as a future upgrade to these dynamic loss maps. Ultimately, all the plots presented in Fig. \ref{fig:lossmaps} demonstrate various potential configurations that open up regions of tune space for utilization during operations, enabling the accommodation of high-intensity beams.

\begin{figure}[H]
    \centering
    \includegraphics[width=\columnwidth]{chapter4/bare.png}
    \caption{Dynamic loss map from ramping the tunes with an interval of $\Delta Q_u=0.005$ in both directions. The directions of scan are from left to right and top to bottom. The results are superimposed in this plot.}
    \label{fig:bare_nocomments}
\end{figure}

\begin{figure}[H]
    \centering
    \includegraphics[width=\columnwidth]{chapter4/bare_comments.png}
    \caption{Dynamic loss map with the corresponding lines from Fig. \ref{fig:rrtd} drawn on top.}
    \label{fig:bare_comments}
\end{figure}


\newpage
\begin{figure}[H]

    \centering
    \begin{subfigure}{.49\textwidth}
      \includegraphics[width=0.98\linewidth]{chapter4/bare.png}
      \caption{Bare machine}
      \label{fig:sfig1}
    \end{subfigure}%
    \hfill
    \begin{subfigure}{.49\textwidth}
      \includegraphics[width=0.98\linewidth]{chapter4/3qx.png}
      \caption{$3Q_x$ Compensation}
      \label{fig:sfig2}
    \end{subfigure}
    \hfill
    \begin{subfigure}{.49\textwidth}
      \includegraphics[width=0.98\linewidth]{chapter4/3qy.png}
      \caption{$3Q_y$ Compensation}
      \label{fig:sfig3}
    \end{subfigure}
    \hfill
    \begin{subfigure}{.49\textwidth}
      \includegraphics[width=0.98\linewidth]{chapter4/3qx_3qy.png}
      \caption{$3Q_x$ and $3Q_y$ Compensation}
      \label{fig:sfig4}
    \end{subfigure}%
    \hfill
    \begin{subfigure}{.49\textwidth}
      \includegraphics[width=0.98\linewidth]{chapter4/3qx_2qxqy.png}
      \caption{$3Q_x$ and $2Q_x+Q_y$ Compensation}
      \label{fig:sfig5}
    \end{subfigure}
    \hfill
    \begin{subfigure}{.49\textwidth}
      \includegraphics[width=0.98\linewidth]{chapter4/3qy_qx2qy.png}
      \caption{$3Q_y$ and $Q_x+2Q_y$ Compensation}
      \label{fig:sfig6}
    \end{subfigure}
    
    \caption{Dynamic loss maps for several configurations of compensation sextupoles}
    \label{fig:lossmaps}
    \end{figure} 
\newpage


\subsection{Static Tune Scans}

Another method for visualizing resonance compensation involves static tune scans. While the loss maps detailed earlier illustrate the dynamic crossing of resonance, an alternative method entails setting the tune to a specific value and assessing the beam survival ratio alongside the beam size over a defined time interval. The beam size is quantified using the Ion Profile Monitor System (IPM) within the ring, and the results are expressed in arbitrary units to indicate a relative effect.

\begin{figure}[H]
    \centering
    \includegraphics[width=\columnwidth]{chapter4/static2turns.png}
    \caption{Static tune scan for bare machine with comparisons between experimental data and SYNERGIA simulations.}
    \label{fig:static2}
\end{figure}

% \begin{figure}[H]
%     \centering
%     \includegraphics[width=\columnwidth]{chapter4/static2turns_comp.png}
%     \caption{Plot for static tune scan for machine with $3Q_x$ compensation including comparison between experimental data and SYNERGIA simulations.}
%     \label{fig:static2_comp}
% \end{figure}

\begin{figure}[H]
    \centering
    \includegraphics[width=\columnwidth]{chapter4/static2turns_ipm.png}
    \caption{Static tune scan with beam survival ratio and IPM data box plots for bare machine at 2 Booster Turns of equivalent intensity.}
    \label{fig:static2_ipm}
\end{figure}

\begin{figure}[H]
    \centering
    \includegraphics[width=\columnwidth]{chapter4/static2turns_comp_ipm_dampersOFF.png}
    \caption{Static tune scan with beam survival ratio and IPM data box plots for machine with $3Q_x$ compensation at 2 Booster Turns of equivalent intensity.}
    \label{fig:static2_ipm_comp}
\end{figure}

\section{Additional Sextupoles for Resonance Compensation}
\label{sec:addsexts}

\begin{equation}
    \begin{bmatrix}
        -|h_{3000}| \cos \psi_{3000} \\
        -|h_{3000}| \sin \psi_{3000} \\
        -|h_{1020}| \cos \psi_{1020} \\
        -|h_{1020}| \sin \psi_{1020} \\
        \end{bmatrix}_{(Bare)}
         =
        \boldsymbol{M}
        \begin{bmatrix}
        k_2^{(sc220)} \\
        k_2^{(sc222)}\\
        k_2^{(sc319)} \\
        k_2^{(sc321)}\\
        \end{bmatrix}
        \label{eq:systemk2s}        
\end{equation}

\begin{figure}[H]
    \centering
    \includegraphics[width=\columnwidth]{chapter4/old_config_h3000.png}
    \caption{Distribution of the $h_{3000}$ term around the ring with individual contributions from each relevant element and the cumulative sum from an arbitrary starting point. This is with existing sextupoles powered at the correct currents to cancel out $h_{3000}$ and $h_{1020}$ simultaneously.}
    \label{fig:h3000oldconfig}
\end{figure}

\begin{figure}[H]
    \centering
    \includegraphics[width=\columnwidth]{chapter4/old_config_h1020.png}
    \caption{Distribution of the $h_{1020}$ term around the ring with individual contributions from each relevant element and the cumulative sum from an arbitrary starting point. This is with existing sextupoles powered at the correct currents to cancel out $h_{3000}$ and $h_{1020}$ simultaneously.}
    \label{fig:h1020oldconfig}
\end{figure}

\begin{equation}
    \begin{bmatrix}
        -|h_{3000}| \cos \psi_{3000} \\
        -|h_{3000}| \sin \psi_{3000} \\
        -|h_{1020}| \cos \psi_{1020} \\
        -|h_{1020}| \sin \psi_{1020} \\
        \end{bmatrix}_{(Bare)}
         =
        \boldsymbol{M}
        \begin{bmatrix}
        k_2^{(sc220)} \\
        k_2^{(sc222)}\\
        k_2^{(sc319)} \\
        k_2^{(sc321)}\\
        k_2^{(1)} \\
        k_2^{(2)}\\
        \end{bmatrix}
        \label{eq:systemadd}
\end{equation}

\begin{figure}[H]
    \centering
    \includegraphics[width=\columnwidth]{chapter4/new_sexts_dx.png}
    \caption{Dispersion function of the Recycler Ring with possible new locations to introduce a pair of sextupole magnets that cancel out $h_{3000}$ and $h_{1020}$ simultaneously. The right y-axis shows the maximum compensation current needed with these new candidates.}
    \label{fig:dxnewsexts}
\end{figure}

\begin{figure}[H]
    \centering
    \includegraphics[width=\columnwidth]{chapter4/new_sexts_h3000.png}
    \caption{Distribution of the $h_{3000}$ term around the ring with individual contributions from each relevant element and the cumulative sum from an arbitrary starting point. This is with the new 620 sextupoles powered at the correct currents to cancel out $h_{3000}$ and $h_{1020}$ simultaneously.}
    \label{fig:h3000newsexts}
\end{figure}

\begin{figure}[H]
    \centering
    \includegraphics[width=\columnwidth]{chapter4/new_sexts_h1020.png}
    \caption{Distribution of the $h_{1020}$ term around the ring with individual contributions from each relevant element and the cumulative sum from an arbitrary starting point. This is with the new 620 sextupoles powered at the correct currents to cancel out $h_{3000}$ and $h_{1020}$ simultaneously.}
    \label{fig:h1020newsexts}
\end{figure}
