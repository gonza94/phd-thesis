\chapter{Compensation of Third-Order Resonances at Low Intensities}
\label{sec:ch4}

\section{Global RDTs and Lattice Model}

\begin{table}[H]
    \centering
    \caption{Corresponding RDTs and spectral lines for each resonance line}
    \begin{tabular}{lcccc}
        \toprule
        \textbf{Resonance Line} & \textbf{Source} & \textbf{RDT} & \textbf{Hor. Spect.} & \textbf{Vert. Spect.} \\
        \midrule
            $3Q_x=76$     & Normal Sextupole    & $h_{3000}$           &  (-2,0)  & -       \\ %[3pt]
           $Q_x+2Q_y=74$   & Normal Sextupole    & $h_{1020}$            & (0,-2) & (-1,-1)       \\ %[3pt]
            $3Q_y=73$     & Skew Sextupole   & $h_{0030}$           & - & (0,-2)        \\ %[3pt]
            $2Q_x+Q_y=75$   & Skew Sextupole    & $h_{2010}$     & (-1,-1) & (-2,0)       \\
        \bottomrule
    \end{tabular}
    \label{tab:rdts}
 \end{table}

\section{Measurement of Third Order RDTs}

\section{Compensation of RDTs}

\section{Optimization of Compensation Currents}

\section{Experimental Verification of Compensation}

\subsection{Dynamic Loss Map}

\subsection{Static Tune Scans}
